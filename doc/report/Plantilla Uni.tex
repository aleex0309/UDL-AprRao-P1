\documentclass[12pt]{article}


\usepackage{graphicx}
\usepackage{fancyhdr}

\usepackage{listings}
\usepackage{hyperref}
\usepackage{amssymb}
\usepackage{enumitem}


\usepackage[utf8]{inputenc}
\usepackage[catalan]{babel}

\usepackage{a4wide}

\usepackage{blindtext}
\renewcommand{\baselinestretch}{1.25}  
\newcommand\crule[3][black]{\textcolor{#1}{\rule{#2}{#3}}}
 
\pagestyle{fancy}
\fancyhf{}
\rhead{Marc Gaspà Joval\\Enginyeria Informàtica}
\lhead{\includegraphics[width=2.5cm]{logoudl.png}}
\setlength{\headheight}{30pt}

\fancyfoot[R]{\thepage}
\rfoot{\thepage}
\renewcommand{\footrulewidth}{0.4pt}
\setlength{\footskip}{25pt}



\begin{document}
\begin{titlepage}
	\newcommand{\HRule}{\rule{\linewidth}{0.5mm}}

	\center

	\includegraphics[width=5cm]{logoudl.png}\\
	\textsc{\Large Enginyeria Informàtica}\\[0.5cm]
	\textsc{\large Raonament i Aprenentatge Automàtic}\\[0.5cm]

	\HRule\\[0.4cm]

	{\huge\bfseries Practica 1}\\[0.4cm]
	{\LARGE Treasure World}
	\HRule\\[1.5cm]

	\textsc{Alexandru Cristian Stoia - Y2386362B\\
		Marc Gaspà Joval - 53923041R}

	\vfill\vfill\vfill
	{\large\today}
\end{titlepage}
\pagebreak

\section{Formula}
\subsection{Variables}
$$ \Gamma = \{ t_{xy}^{t-1}, t_{xy}^{t+1}, S1_{xy}, S2_{xy}, S3_{xy}   \} $$

\subsubsection{Description}
We define two variables for each position $xy$ on the map:
\begin{itemize}[label={}]
	\item $t_{xy}$, which is true if the treasure is located at position $xy$.
	\item $S_{xy}^i$, which is true if sensor $i\in{1,2,3}$ is activated at position $xy$.
\end{itemize}

\subsubsection{Future and past times}
For the variable $t_{xy}$, we have two versions: one for the past and one for the future. The past version indicates what we already know, and the future version is used to update the past version.\\\\
To convert from the past version to the future version, we use the following implication:
$$ t_{xy}^{t-1} \rightarrow t_{xy}^{t+1} $$
Note that in the exercise, we are given a function called \textbf{addLastFutureClausesToPastClauses()}, which performs the conversion from the future version to the past version.

\pagebreak
\subsection{Implications}
For each sensor we add the necessary implications to encode all
positions of the map

\subsubsection{Sensor 1}
If \textbf{sensor 1} is activated in $xy$ means the treasure is located in one of the five cells $\{(x, y- 1), (x, y), (x, y + 1), (x- 1, y), (x +
	1, y)\}$. So, it is NOT in any other cells of the world.

\begin{tabbing}
	$ \forall xy \in \{N*N\}: $\\
	\hspace{1em} $ \forall x'y' \in \{"\textrm{4 corners adjacent to } xy"\}\cup
		\{"\textrm{all other but 9 adjacent to }xy"\}  :$\\
	$ \hspace{3em} S1_{xy} \rightarrow \lnot t_{x'y'}^{t+1}$
\end{tabbing}

\subsubsection{Sensor 2}
If \textbf{sensor 2} is activated the treasure is located in one of the four cells $\{(x + 1, y + 1), (x + 1, y - 1), (x - 1, y -
	1), (x - 1, y + 1)\}$. So, it is NOT in any other cells of the world.
\begin{tabbing}
	$ \forall xy \in \{N*N\}: $\\
	\hspace{1em} $ \forall x'y' \in \{"\textrm{5 cells forming a cross arround } xy"\}\cup
		\{"\textrm{all other but 9 adjacent to }xy"\}  :$\\
	$ \hspace{3em} S2_{xy} \rightarrow \lnot t_{x'y'}^{t+1}$
\end{tabbing}

\subsubsection{Sensor 3}
If \textbf{sensor 3} is activated means the treasure is NOT located in any of the nine cells of the filled square centered around $xy$:
\begin{tabbing}
	$ \forall xy \in \{N*N\}: $\\
	\hspace{1em} $ \forall x'y' \in \{"\textrm{9 cells arround } xy"\}:$\\
	$ \hspace{3em} S3_{xy} \rightarrow \lnot t_{x'y'}^{t+1}$
\end{tabbing}

\end{document}